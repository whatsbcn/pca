\section{Introducci\'{o}n}

Para realizar los profiles del programa hemos utilizado la herramienta gprof.
Hemos escogido esta porque, aunque los resultados pueden diferir, nos da los
valores absolutos en la aplicaci\'{o}n ejecutada y no en el sistema. Para
ejecutar profile hemos a\~{n}adido los flags \texttt{-g} \texttt{-p} en la compilaci\'{o}n. Una vez
modificado el c\'{o}digo, ejecutabamos gprof y guardabamos los valores obtenidos. 

Hemos utilizado los scripts principalmente para preparar el entorno de
pruevas, y para cambiar de una optimizaci\'{o}n a otra. No hemos utilizado scripts
para comprobar las diferencias de rendimiento entre ejecuciones, ya que hicios
que se printase por salida de error, el elapsed time de cada ejecuci\'{o}n.
Teniendo esto, solo nos era necesario ejecutar y comparar el tiempo entre ejecuciones.

Para comprobar que un fichero de salida era correcto, hemos utilizado la
herramienta md5sum. 

Nos hemos dedicado a optimizar las funciones, empezando por la que consume m\'{a}s
tiempo y acabando en la que menos.

Hemos realizado la pr\'{a}ctica en los mismos ordenadores donde hac\'{i}amos los
laboratorios. Los archivos de programa y de entrada, los guardaba antes de
empezar el directorio tmp para evitar latencias debido al transporte de la
red.

% vim: filetype=tex tw=75
