\section{Introducci\'{o}n}

Este informe analiza y explica las optimizaciones que se han llevado a cabo en
el programa FTDock.

El trabajo realizado consiste en aplicar optimizaciones sobre el c\'{o}digo
base, analizando en cada paso cual es la parte de la aplicaci\'{o}n que se va a
mejorar y como se puede puede conseguir la mejora.

Adem\'{a}s de optimizar el c\'{o}digo del FTDock se han generado una serie de
scripts para hacer m\'{a}s f\'{a}cil el proceso de instalaci\'{o}n de la
aplicaci\'{o}n, la compilaci\'{o}n de las diversas versiones del c\'{o}digo y
la ejecuci\'{o}n de las pruebas. Junto a este informe se ha hecho entrega de
estos scripts, todos los archivos necesarios para probar las optimizaciones y
un archivo de ayuda que explica como utilizar los scripts.

Este documento se estructura de la siguiente forma: en la secci\'{o}n de
metolog\'{i}a se explican los pasos que se han seguido durante el proceso de
optimizaci\'{o}n del c\'{o}digo, as\'{i} como las herramientas utilizadas. En
la secci\'{o}n de optimizaciones se explican las optimizaciones realizadas,
justificando las modificaciones y mostrando lo que se gana con ellas. El
documento acaba con las conclusiones que se extraen de este trabajo.

N\'{o}tese que, por motivos de espacio, en la secci\'{o}n de optimizaciones no
se ha adjuntado todo el c\'{o}digo de cada versi\'{o}n sin\'{o} las partes
necesarias para explicar que se ha hecho.

% vim: filetype=tex tw=75
