\section{Metodolog\'{i}a}

La metodolog\'{i}a que se debe utilizar para llevar a cabo optimizaciones es un
proceso iterativo que incluye los siguientes pasos en cada iteraci\'{o}n:

\begin{itemize}
\item Identificar la parte del c\'{o}digo que consume m\'{a}s tiempo
\item Realizar una optimizaci\'{o}n
\item Comprobar que la optimizaci\'{o}n mejora el rendimiento del programa
\end{itemize}

Utilizando la herramienta \texttt{gprof} se han identificado las partes del
c\'{o}digo original que consumen la mayor\'{i}a del tiempo de ejecuci\'{o}n.
La salida proporcionada por dicha herramienta es la siguiente:


Los resultados dejan claros los dos objetivos de las optimizaciones: la
funci\'{o}n \texttt{electric\_field} y la funci\'{o}n \texttt{pythagoras}.

La gran diferencia en tiempo de ejecuci\'{o}n que existe entre estas dos
funciones y todas las dem\'{a}s hace que el primer paso del proceso iterativo
descrito anteriormente no tenga sentido, ya que es casi imposible que cualquier
otra funci\'{o}n pase a dominar el tiempo de ejecuci\'{o}n de la aplicaci\'{o}n.
En caso de que pase significa que la mejora sobre \texttt{electric\_field} y
\texttt{pythagoras} es tan grande que el objetivo de optimizar FTDock est\'{a}
m\'{a}s que cumplido.

Se han utilizado scripts principalmente para preparar el entorno de pruebas y
para pasar r\'{a}pidamente de una optimizaci\'{o}n a otra. No se han utilizado
scripts para comprobar las diferencias de rendimiento entre ejecuciones. Para
comprobar que las salidas de los diferentes versiones son correctas se ha
utilizado la herramienta \texttt{md5sum}.

El trabajo se ha realizado en los ordenadores del laboratorio de la asignatura,
instalando la aplicaci\'{o}n en el directorio \texttt{/tmp} para evitar el uso
del sistema de ficheros NFS.

% vim: filetype=tex tw=75
