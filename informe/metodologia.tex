\section{Metodolog\'{i}a}

La metodolog\'{i}a que se debe utilizar para llevar a cabo optimizaciones es un
proceso iterativo que incluye los siguientes pasos en cada iteraci\'{o}n:

\begin{itemize}
\item Identificar la parte del c\'{o}digo que consume m\'{a}s tiempo
\item Realizar una optimizaci\'{o}n
\item Comprobar que la optimizaci\'{o}n mejora el rendimiento del programa
\end{itemize}

Utilizando la herramienta \texttt{gprof} se han identificado las partes del
c\'{o}digo original que consumen la mayor\'{i}a del tiempo de ejecuci\'{o}n.
La salida proporcionada por dicha herramienta es la siguiente:


Los resultados dejan claros los dos objetivos de las optimizaciones: la
funci\'{o}n \texttt{electric\_field} y la funci\'{o}n \texttt{pythagoras}.

Para esta aplicaci\'{o}n en particular, la gran diferencia en tiempo de
ejecuci\'{o}n que existe entre estas dos funciones y todas las dem\'{a}s hace
que el primer paso del proceso iterativo descrito anteriormente d\'{e} siempre
como objetivo de las optimizaciones las mismas funciones, ya que es casi
imposible que cualquier otra funci\'{o}n pase a dominar el tiempo de
ejecuci\'{o}n de la aplicaci\'{o}n. En caso de que pasara significar\'{i}a que
la mejora sobre \texttt{electric\_field} y \texttt{pythagoras} es tan grande
que el objetivo de optimizar FTDock est\'{a} m\'{a}s que cumplido. Por esta
raz\'{o}n y por limitaciones de espacio no se mestra la salida de \texttt{gprof}
de cada optimizaci\'{o}n.

Se han utilizado scripts principalmente para preparar el entorno de pruebas y
para pasar r\'{a}pidamente de una optimizaci\'{o}n a otra. No se han utilizado
scripts para comprobar las diferencias de rendimiento entre ejecuciones. Para
comprobar que las salidas de los diferentes versiones son correctas se ha
utilizado la herramienta \texttt{md5sum}.

Para cada c\'{o}digo se han lanzado tres tests distintos. Los argumentos de
dichos tests son, respectivamente:

\begin{lstlisting}[]
    ./ftdock -static 2pka.parsed -mobile 5pti.parsed
    ./ftdock -static 1hba.parsed -mobile 5pti.parsed
    ./ftdock -static 4hhb.parsed -mobile 5pti.parsed
\end{lstlisting}

El trabajo se ha realizado en los ordenadores del laboratorio de la asignatura,
instalando la aplicaci\'{o}n en el directorio \texttt{/tmp} para evitar pasar
por el sistema de ficheros NFS y as\'{i} no distorsionar los tiempos de
ejecuci\'{o}n.

% vim: filetype=tex tw=75
