\section{Conclusiones}

El trabajo ha servido para poner en pr\'{a}ctica muchos de los recursos
explicados durante el curso y se ha comprobado que es muy beneficioso aplicar
optimizaciones sobre una aplicaci\'{o}n real. 

El hecho de tener que trabajar sobre una aplicaci\'{o}n compleja ya hecha ha
servido para aprender a analizar un c\'{o}digo que est\'{a} formado por muchos
m\'{o}dulos y muchas funciones. Esto a\~{n}ade bastante dificultad al problema
al tener que analizar distintas partes para ver como est\'{a} estructurado el
programa. En ese sentido el profiling es de gran ayuda, ya que permite empezar
a optimizar la aplicaci\'{o}n sin conocer muy bien la estructura del c\'{o}digo
ni saber como funciona realmente la aplicaci\'{o}n.

Aplicar vectorizaci\'{o}n suele ser muy beneficioso para el rendimiento de una
aplicaci\'{o}n. Lamentablemente algunas partes del programa no se han podido
optimizar tanto como hubiera sido deseable ya que las intr\'{i}nsecas necesarias
para hacerlo no existen. Otra lecci\'{o}n aprendida es que vectorizar funciones
no siempre implica mejorar el rendimiento ya que el coste de cargar y descargar
los registros vectoriales es elevado.

El trabajo y la asignatura en genral muestra lo importante que es el compilador
para crear c\'{o}digo eficiente y lo importante que es conocer bien la
arquitectura en la que se va a ejecutar un c\'{o}digo para obtener un mayor
rendimiento. Hay que a\~{n}adir que de la asignatura se ha aprendido que el
buen hacer de un procesador no viene dado solo por lo r\'{a}pido que
ejecuta las instrucciones sin\'{o} que el conjunto de instrucciones que soporta
es igualmente importante.

% vim: filetype=tex tw=75
